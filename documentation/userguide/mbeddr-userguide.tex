
\documentclass[12pt,parskip,a4paper,final,oneside,numbers=noenddot,headsepline]{scrbook}


\usepackage{amssymb} 
\usepackage{soul}   
\usepackage[pdftex]{graphicx}
\usepackage[utf8]{inputenc} 
\usepackage{comment}
\usepackage{framed}
\usepackage{wrapfig}
\usepackage{color}
\usepackage[T1]{fontenc} 
\usepackage{listings}

\definecolor{shadecolor}{rgb}{.8,.8,.8}

\newcommand{\greyboxnote}[1]{\begin{shaded}
\noindent\textbf{Note: }{#1}\end{shaded}\vspace{-5mm}}

\usepackage{verbatim}
	 
\newenvironment{code}
    {\noindent \footnotesize \verbatim
    \vspace{-4mm}\addtolength{\leftskip}{7.5mm}} 
    {\endverbatim \vspace{-4mm} \normalsize}

 

\newcommand{\changefont}[3]{\fontfamily{#1}\fontseries{#2}\fontshape{#3}\selectfont}

     

\newcommand{\mynote}[2]{\textbf{{#1}(}\textit{{#2}}\textbf{)}}
\newcommand\todo[1]{{\color{red}\mynote{TODO}{#1}}} 
\newcommand\dan[1]{\mynote{DAN}{#1}} 
\newcommand\markus[1]{\mynote{MARKUS}{#1}} 
\newcommand\bernd[1]{\mynote{BERND}{#1}} 
\newcommand\bernhard[1]{\mynote{BERNHARD}{#1}} 

\newcommand{\fig}[1]{Fig.~\ref{#1}}
\newcommand{\sect}[1]{Section~\ref{#1}}

\newcommand{\ic}[1]{\changefont{cmtt}{m}{n}{#1}\normalfont}  % inline code
\newcommand{\lcr}[1]{\changefont{cmtt}{m}{n}{#1}\normalfont} % language
\newcommand{\keystroke}[1]{\changefont{cmtt}{m}{n}{#1}\normalfont} % language
\newcommand{\intention}[1]{\emph{#1}} % language

  
\begin{document}

\vspace{50mm}


\begin{center} 
  \includegraphics{logoWithSlogan.png}

\vspace{5mm}
This document is part of the \\ 
mbeddr project at \ic{http://mbeddr.com}. 

\vspace{5mm}
This document is licensed under the \\ 
Eclipse Public License 1.0 (EPL).
\end{center}

\vfill

\huge
\changefont{phv}{bc}{n}mbeddr C User Guide\normalfont
\normalsize

This document focuses on the C programmer who wants to exploit the benefits of
the extensions to C provided by mbeddr out of the box. We assume that you have
some knowledge of regular C (such asK\&R C, ANSI C or C99). We also assume that
you realize some of the shortfalls of C and are "open" to the improvements in
mbeddr C. The main point of mbeddr C is the ability to extend C with
domain-specific concepts such as state machines, components, or whatever you
deem useful in your domain. We have also removed some of the "dangerous"
features of C that are often prohibited from use in real world projects.

This document does not discuss how to develop new languages or extend existing
languages. We refer to the \emph{Extension Guide} instead. It is available from
\ic{http://mbeddr.com}.



\setcounter{tocdepth}{2}
\tableofcontents % Inhaltsverzeichnis
 
%\listoffigures % Abbildungsverzeichnis 

\mainmatter % Hauptkapitel

 
\input{body.ltx} % Abstract 


\end{document}